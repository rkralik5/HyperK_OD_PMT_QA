\documentclass[12pt,a4paper]{article}
\usepackage{lineno}
\usepackage[T1]{fontenc}
\usepackage{graphicx}
\usepackage{hyperref}
\usepackage{caption}
\usepackage{mathptmx}
\usepackage[absolute,overlay]{textpos}
\usepackage{fancybox}
\usepackage{multirow}
\usepackage[dvipsnames,table,xcdraw]{xcolor}
\usepackage{amsmath, amsthm, amssymb,amsfonts}
\usepackage{float}
\usepackage{fullpage}
\usepackage{units}
\usepackage{xspace}
\usepackage{caption}
\usepackage{subcaption}
\usepackage{array}
\usepackage{tikz}
\usepackage{placeins}  % for FloatBarrier
\usepackage[nottoc,notlot,notlof]{tocbibind}
\usetikzlibrary{decorations.pathreplacing}
\usepackage{authblk} % To add authors' affiliations

%%% Code to enter C++ code
\usepackage{listings}
\linenumbers % Include line numbers
\lstset{language=C++,
        basicstyle=\ttfamily,
        keywordstyle=\color{blue}\ttfamily,
        stringstyle=\color{red}\ttfamily,
        commentstyle=\color{green}\ttfamily,
        morecomment=[l][\color{magenta}]{\#}
}

\usepackage{hyperref}
\hypersetup{
    colorlinks,
    citecolor=blue!90!black,
    filecolor=black,
    linkcolor=blue!50!black,
    urlcolor=blue!50!black
}

%%% Set personal note commands
\newcommand{\todo}[1]{\textcolor{red!90!black}{TO DO: \textit{#1}}}
\newcommand{\note}[1]{\textcolor{green!70!black}{COMMENT: \textit{#1}}}
\newcommand{\inches}{$''$\xspace}

\author{Róbert Králik for the FD3 group}
%\affil[1]{King's College London, UK}
\title{\textbf{Shift overview for the Quality Assurance of the 3'' OD PMTs\\ \vspace*{5mm}
\Large{Technical Note}}}
\date{\today}

\begin{document}
\maketitle
%\thispagestyle{empty} % Remove page number from title page

\tableofcontents

\vspace*{1cm}

\section{Introduction}

Each Hyper-Kamiokande Outer Detector (OD) photomultiplier tube (PMT) will undergo quality assurance (QA) to assess performance and ensure reliable long-term operation. The OD PMT QA program consists of a series of electrical and optical tests performed on PMTs after delivery and prior to their integration into PMT+WLS units and installation.

This technical note describes the shift work associated with the OD PMT QA, with a particular focus on personnel roles, shift organisation, training requirements, and safety considerations. It is intended to support safety review and shift planning for the QA task.

A detailed description of the QA setup, procedures, performance metrics, timeline, and acceptance criteria is outside the scope of this document and will be covered in dedicated OD PMT QA documentation. Interfaces with other OD activities, including PMT delivery, storage, and assembly, are described in the corresponding OD installation procedures, logistics, and instructions technical note~\cite{ODAssemblyTN}.

\section{Shift work}\label{secShiftWork}

The OD PMT QA work will take place in a ground-floor room at the University of Toyama. PMTs will be stored before and after QA in a separate ground-floor storage area at the same site. QA will be carried out in the same room as the PMT+WLS unit assembly, within a clearly marked and optically separated area equipped with two dark boxes and the required readout and control equipment.

PMTs will be transported between the storage and QA areas at ground level, typically using a trolley to reduce manual handling. As the QA space is shared with other OD activities, including PMT assembly, coordination is required to maintain clear access paths and to separate tasks where needed. Further details of room layout, storage, transportation, and assembly are provided in the OD installation technical note~\cite{ODAssemblyTN}.

\subsection{Tasks and responsibilities}
During QA shifts, shifters will handle PMTs and operate the QA test setup. This includes transporting PMTs between storage and the QA workspace, loading and unloading them into dark boxes, and connecting and disconnecting high-voltage (HV) and signal cables. PMTs will be moved in boxes of six, each weighing up to $\sim15$\,kg, while individual PMTs are lightweight. Transport will be assisted by carts or trolleys. Handling of PMT transport boxes during delivery may also be required, depending on final logistics.

Shifter tasks include opening and closing dark boxes, retrieving and positioning PMTs on PMT stands, performing basic visual inspections, scanning and logging PMTs with a barcode scanner, and operating HV and data acquisition systems. These activities involve repetitive handling and working in bent positions. A photo of a PMT on a PMT stand inside a dark box is shown in Fig.~\ref{fig:DarkBoxExample}.

\begin{figure}[hbtp!]
    \centering
    \includegraphics[width=0.5\textwidth]{figures/QA_DarkBoxExample.jpg}
    \caption{A single PMT mounted on a PMT stand inside one of the QA dark boxes. The orange cablle is an optical fibre used to deliver light from an LED pulser (not connected here) to the PMT photocathode and is connected to a cap that is covering and protecting the PMT face.}
    \label{fig:DarkBoxExample}
\end{figure}

Shifters will also handle optical fibres, LEDs, and data acquisition hardware such as ADCs, high-voltage power supplies, low-voltage function generator, and a control computer.

The QA shifts may also involve external workers, such as students from the University of Toyama. These workers could perform routine tasks such as transporting PMTs, loading and unloading of PMTs into dark boxes, and basic visual inspections, under the supervision of trained shifters and/or the shift leader.

\subsection{Training and supervision}
Shifters will receive training in manual handling of PMTs, working with fragile equipment, and basic awareness of high- and low-voltage equipment. Training will also cover scanning and logging PMTs, handling optical fibres and LEDs, operating data acquisition hardware and software, and using carts or trolleys for transport.

Routine QA tasks, such as loading and unloading PMTs, visual inspections, and data acquisition, can be performed independently by trained shifters. An expert will be available for guidance during initial setup, troubleshooting, final data analysis and approval, or any non-routine adjustments to the QA equipment.

All shifters must be aware of potential hazards, including repetitive handling, bent postures, and interaction with HV equipment, and must follow established safety procedures.

External workers, if hired, will receive the same training as regular shifters, including manual handling, awareness of fragile equipment, and basic high-voltage safety. They will perform tasks only under supervision and will not carry out non-routine adjustments to the QA setup.

\subsection{Schedule and workload}
QA shifts are planned to last up to 8 hours per working day. Shifters will work in pairs to safely handle PMTs, including loading, unloading, transporting between storage and the QA workspace, and performing QA measurements. At least one experienced shift leader will be present near or in Toyama at all times. The total QA campaign is expected to span several weeks, depending on the PMT delivery schedule and testing rate.

\section{Safety considerations}
The QA work involves working with PMTs and associated equipment, and includes potential hazards related to manual handling, electrical connections, and workspace coordination. Key safety considerations for shifters are summarized below:
\begin{itemize}
    \item \textbf{Manual handling:} PMTs will be handled in boxes of six, weighing up to $\sim15$\,kg. Individual PMTs are lightweight. Shifters must use proper lifting techniques and, where possible, use trolleys or carts to transport PMT boxes between storage and the QA workspace. Training in manual handling will be provided.
    \item \textbf{Working postures:} Loading and unloading PMTs into dark boxes will require bent postures. Shifters should take regular breaks and limit the work in a bent position as much as possible.
    \item \textbf{Fragile equipment:} PMTs are fragile and must be handled with care to avoid damage. Shifters should follow established procedures for handling, loading, and unloading PMTs, and report any damage or issues immediately. Shifters will wear gloves and safety goggles when handling PMTs to protect against accidental breakage.
    \item \textbf{Electrical safety:} The QA setup includes high-voltage power supplies and low-voltage function generators. Shifters must be trained in basic electrical safety, avoid contact with live circuits, and follow established procedures for connecting and disconnecting HV cables. The HV power supplies will be automatically turned off when not in use and the QA area will be clearly marked with appropriate signage. The splitters for PMT signal and HV will be enclosed in a protective casing to prevent accidental contact with high-voltage parts and located outside easy reach of shifters. 
    \item \textbf{Workspace coordination:} The QA area is shared with other OD activities, including PMT assembly. Clear access paths must be maintained, and tasks should be coordinated to avoid interference or hazards.
    \item \textbf{Emergency procedures:} Shifters must be familiar with emergency procedures, including evacuation routes, first aid locations, and contact information for emergency services.
\end{itemize}

%\FloatBarrier
\bibliographystyle{unsrturl}
\bibliography{ODPMTQATechNoteLiterature}
\end{document}