\documentclass[12pt,a4paper]{article}
\usepackage{lineno}
\usepackage[T1]{fontenc}
\usepackage{graphicx}
\usepackage{hyperref}
\usepackage{caption}
\usepackage{mathptmx}
\usepackage[absolute,overlay]{textpos}
\usepackage{fancybox}
\usepackage{multirow}
\usepackage[dvipsnames,table,xcdraw]{xcolor}
\usepackage{amsmath, amsthm, amssymb,amsfonts}
\usepackage{float}
\usepackage{fullpage}
\usepackage{units}
\usepackage{xspace}
\usepackage{caption}
\usepackage{subcaption}
\usepackage{array}
\usepackage{tikz}
\usepackage{placeins}  % for FloatBarrier
\usepackage[nottoc,notlot,notlof]{tocbibind}
\usetikzlibrary{decorations.pathreplacing}
\usepackage[colorinlistoftodos]{todonotes}
\usepackage{authblk} % To add authors' affiliations

%%% Code to enter C++ code
\usepackage{listings}
\linenumbers % Include line numbers
\lstset{language=C++,
        basicstyle=\ttfamily,
        keywordstyle=\color{blue}\ttfamily,
        stringstyle=\color{red}\ttfamily,
        commentstyle=\color{green}\ttfamily,
        morecomment=[l][\color{magenta}]{\#}
}

\usepackage{hyperref}
\hypersetup{
    colorlinks,
    citecolor=blue!90!black,
    filecolor=black,
    linkcolor=blue!50!black,
    urlcolor=blue!50!black
}

\newcommand{\inches}{$''$\xspace}

\author{Róbert Králik for the FD3 group}
%\affil[1]{King's College London, UK}
\title{\textbf{Shift overview for the Quality Assurance of the 3\inches OD PMTs\\ \vspace*{5mm}
\Large{Technical Note}}}
\date{\today}

\begin{document}
\maketitle
%\thispagestyle{empty} % Remove page number from title page

\tableofcontents

\vspace*{1cm}

\section{Introduction}

Each Hyper-Kamiokande Outer Detector (OD) photomultiplier tube (PMT) will undergo quality assurance (QA) to assess performance and ensure reliable long-term operation. The OD PMT QA program consists of a series of electrical and optical tests performed on PMTs after delivery and prior to their integration into PMT+WLS units and installation.

This technical note describes the shift work associated with the OD PMT QA, with a particular focus on personnel roles, shift organisation, training requirements, and safety considerations. It is intended to support safety review and shift planning for the QA task.

A detailed description of the QA setup, procedures, performance metrics, timeline, and acceptance criteria is outside the scope of this document and will be covered in the dedicated OD PMT QA documentation. Interfaces with other OD activities, including PMT delivery, storage, and assembly, are described in the corresponding OD installation procedures, logistics, and instructions technical note~\cite{ODAssemblyTN}.

\section{Shift work}\label{secShiftWork}

The OD PMT QA work will take place in a ground-floor room at the University of Toyama. PMTs will be stored before and after QA in a separate ground-floor storage area at the same site. QA will be carried out in the same room as the PMT+WLS unit assembly, within a clearly marked and optically separated area equipped with two dark boxes and the required readout and control equipment.

PMTs will be transported between storage and QA areas at ground level, typically using a trolley to reduce manual handling. The shifters will keep all the necessary equipment, including carts or trolleys, in designated areas within the OD rooms. As the QA space is shared with other OD activities, including PMT assembly, coordination is required to maintain clear access paths and separate tasks where needed. Further details of room layout, storage, transportation, and assembly are provided in the OD installation technical note~\cite{ODAssemblyTN}.

\subsection{Tasks and responsibilities}
QA shifts involve the following tasks:
\begin{enumerate}
    \item Transport of PMTs between storage and the QA workspace using carts or trolleys
    \begin{itemize}
        \item PMTs will be in transport boxes containing six PMTs each
        \item Each box weighs up to $\sim13$\,kg
    \end{itemize}
    \item Loading and unloading PMTs into dark boxes and positioning them on PMT stands
    \begin{itemize}
        \item Individual PMTs are lightweight
        \item A photo of a PMT on a stand inside a dark box is shown in Fig.~\ref{fig:DarkBoxExample}
    \end{itemize}
    \item Connecting and disconnecting PMT cables from high-voltage/signal splitters
    \item Operation of high-voltage power supplies, analog-to-digital converters, low-voltage function generator, and a control computer
    \item Handling of optical fibres and LEDs
    \item Basic visual inspection of PMTs
    \item Scanning and logging PMTs using a barcode scanner
    \item Operating data acquisition software, recording and checking QA data, logging results in the database
    \item Deciding on the acceptance or rejection of PMTs
    \item (Potentially) handling PMT transport boxes during delivery
\end{enumerate}

\begin{figure}[hbtp!]
    \centering
    \includegraphics[width=0.5\textwidth]{figures/QA_DarkBoxExample.jpg}
    \caption{A single PMT mounted on a PMT stand inside one of the QA dark boxes. The orange cablle is an optical fibre used to deliver light from an LED pulser (not connected here) to the PMT photocathode and is connected to a cap that is covering and protecting the PMT face.}
    \label{fig:DarkBoxExample}
\end{figure}

QA shifts involve three categories of personnel: experts, regular shifters, and, potentially, external workers. Their roles and responsibilities are defined below.

\subsubsection*{Experts}
The QA expert is responsible for the overall operation and integrity of the QA process. The expert will:
\begin{itemize}
    \item Oversee the initial setup of the QA system and approve any non-routine changes to the setup or procedures
    \item Perform training of shifters in the beginning of the shift block
    \item Provide guidance during troubleshooting and abnormal situations
    \item Perform or approve final data analysis and decide on PMT acceptance
    \item Ensure that QA results are correctly uploaded to the database
\end{itemize}
The expert does not need to be physically present for routine QA shifts but must be available locally or remotely for consultation. The experts can help and perform regular QA shift when necessary. The expert responsibilities may be shared with other OD tasks if this proves viable.

\subsubsection*{Regular shifters}
QA shifts are performed by two regular shifters working together. At least one of the shifter should be experienced from a previous QA shift.

Regular QA shifters are trained collaborators certified to perform routine QA tasks independently. During QA shifts, they will:
\begin{itemize}
    \item Transport PMTs between storage and the QA workspace using carts or trolleys
    \item Load and unload PMTs into dark boxes and position them on PMT stands
    \item Scan and log PMTs using a barcode scanner
    \item Perform basic visual inspection of PMTs
    \item Connect PMT cables to HV/signal splitters
    \item Operate the HV and data acquisition software and hardware
    \item Perform basic analysis of results and upload them to the database
\end{itemize}

\subsubsection*{External workers}
The QA shifts may also involve external workers, such as students from the University of Toyama. These workers could perform routine tasks such as
\begin{itemize}
    \item transporting PMTs
    \item loading and unloading of PMTs into dark boxes
    \item basic visual inspections
\end{itemize}
External workers should work under supervision of trained shifters and/or the shift leader.

\subsection{Training and supervision}
All shifters, including external workers, will receive training by an expert at the beginning of the shift block. Training will include manual handling of PMTs, working with fragile equipment, and basic awareness of high- and low-voltage equipment. It will also cover scanning and logging PMTs, handling optical fibres and LEDs, operating data acquisition hardware and software, and using carts or trolleys for transport.

If all shifters are experienced shifters, this training can be omitted. In any case, the experts must ensure that all the shifters are aware of potential hazards, including repetitive handling, bent postures, and interaction with HV equipment, and must follow established safety procedures.

\subsection{Schedule and workload}
QA shifts are planned to last up to 8 hours per working day. The shifters will work in pairs to safely handle PMTs, including loading, unloading, transporting between storage and the QA workspace, and performing QA measurements. At least one experienced shift leader will be present near or in Toyama at all times.

The total number of QA shifts and the detailed schedule depend on the PMT delivery schedule. The preliminary plan assumes the start of PMT deliveries at the end of July 2026, with approximately 300–400 PMTs delivered monthly thereafter, as shown in Fig.~\ref{fig:PMTDeliverySchedule}. We plan to test all PMTs within a month of delivery. The QA campaign is expected to continue until the completion of PMT deliveries, which is currently planned for April 2027.

\begin{figure}
    \centering
    \includegraphics[width=0.8\linewidth]{figures/PMTDeliverySchedule_Jan2026.png}
    \caption{Preliminary schedule of the PMT delivery as of January 2026.}
    \label{fig:PMTDeliverySchedule}
\end{figure}

\section{Safety considerations}
The QA work involves working with PMTs and associated equipment, and includes potential hazards related to manual handling, electrical connections, and workspace coordination. Key safety considerations for shifters are summarized below:
\begin{itemize}
    \item \textbf{Manual handling:} PMTs will be handled in boxes of six, weighing up to $\sim13$\,kg. Individual PMTs are lightweight. Shifters must use proper lifting techniques and, where possible, use trolleys or carts to transport PMT boxes between storage and the QA workspace. Training in manual handling will be provided.
    \item \textbf{Working postures:} Loading and unloading PMTs into dark boxes will require bent postures and repetitive bending. Shifters should take regular breaks as they see fit and the shifts will be optimized to limit the work in a bent position as much as possible.
    \item \textbf{Fragile equipment:} PMTs are fragile and must be handled with care to avoid damage. Shifters should follow established procedures for handling, loading, and unloading PMTs, and report any damage or issues to the expert immediately. Shifters will wear gloves and safety goggles when handling PMTs to protect against accidental breakage.
    \item \textbf{Electrical safety:} The QA setup includes high-voltage power supplies and low-voltage function generators. Shifters must be trained in basic electrical safety, avoid contact with live circuits, and follow established procedures for connecting and disconnecting HV cables. The HV power supplies will be automatically turned off when not in use and the QA area will be clearly marked with appropriate signage. The splitters for PMT signal and HV will be enclosed in a protective casing to prevent accidental contact with high-voltage parts and located outside easy reach of shifters. 
    \item \textbf{Workspace coordination:} The QA area is shared with other OD activities, including PMT assembly. Shifters must maintain clear access paths when moving equipment, and tasks should be coordinated to avoid interference or hazards.
    \item \textbf{Emergency procedures:} Shifters must be familiar with emergency procedures, including evacuation routes, first aid locations, and contact information for emergency services.
\end{itemize}

%\FloatBarrier
\bibliographystyle{unsrturl}
\bibliography{ODPMTQATechNoteLiterature}
\end{document}